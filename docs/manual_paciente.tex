\documentclass[a4paper,12pt]{article}
\usepackage[utf8]{inputenc}
\usepackage[spanish]{babel}
\usepackage{geometry}
\usepackage[hidelinks]{hyperref}
\usepackage{xcolor}
\usepackage{titlesec}
\usepackage{graphicx}

% Configuración de márgenes
\geometry{top=3cm, bottom=3cm, left=2.5cm, right=2.5cm}

% Estilo de títulos
\titleformat{\section}{\color{green!60!black}\normalfont\Large\bfseries}{\thesection}{1em}{}
\titleformat{\subsection}{\color{green!40!black}\normalfont\large\bfseries}{\thesubsection}{1em}{}

\title{\textbf{Guía del Paciente: Su Asistente de Salud IA}}
\author{Sistema de Telemedicina}
\date{\today}

\begin{document}

\maketitle
\tableofcontents
\newpage

\section{Bienvenido}
Este asistente virtual está diseñado para acompañarle en su proceso de recuperación y control de salud. Puede interactuar con él mediante texto o notas de voz a través de Telegram.

\section{Gestión de Citas}

\subsection{Agendar una Cita}
Para programar una nueva consulta o revisión:
\begin{itemize}
    \item \textbf{Comando:} \texttt{/cita [DD/MM] [HH:MM] [Motivo]}
    \item \textbf{Ejemplo:} \texttt{/cita 25/10 15:30 Revisión general}
\end{itemize}

\subsection{Ver Mis Citas}
Para consultar sus próximas citas agendadas:
\begin{itemize}
    \item \textbf{Comando:} \texttt{/mis\_citas}
\end{itemize}

\section{Recordatorios de Medicación}

\subsection{Configurar Alarma}
El asistente puede recordarle tomar sus medicamentos diariamente:
\begin{itemize}
    \item \textbf{Comando:} \texttt{/recordatorio [HH:MM] [Mensaje]}
    \item \textbf{Ejemplo:} \texttt{/recordatorio 08:00 Tomar pastilla de la tensión}
\end{itemize}

\subsection{Borrar Recordatorios}
Para eliminar todas las alarmas configuradas:
\begin{itemize}
    \item \textbf{Comando:} \texttt{/borrar\_recordatorios}
\end{itemize}

\section{Herramientas de Asistencia}

\subsection{Análisis de Documentos}
Si tiene un resultado de laboratorio o informe médico en PDF:
\begin{enumerate}
    \item Envíe el archivo PDF al chat.
    \item El asistente lo leerá y le explicará el contenido en lenguaje sencillo.
\end{enumerate}

\subsection{Traducción}
Para traducir documentos médicos o textos a español:
\begin{itemize}
    \item \textbf{Comando:} \texttt{/traducir [texto o nombre de archivo]}
\end{itemize}

\subsection{Interacción por Voz}
Puede enviar notas de voz en cualquier momento. El asistente:
\begin{itemize}
    \item Transcribirá su mensaje.
    \item Procesará su consulta.
    \item Le responderá con texto y también con una nota de voz.
\end{itemize}

\subsection{Cambiar Idioma}
Si prefiere interactuar en otro idioma (para la voz):
\begin{itemize}
    \item \textbf{Comando:} \texttt{/idioma [es/en/fr/pt]}
    \item \textbf{Ejemplo:} \texttt{/idioma en} (para inglés)
\end{itemize}

\section{Ayuda}
Si olvida los comandos, siempre puede solicitar ayuda:
\begin{itemize}
    \item \textbf{Comando:} \texttt{/ayuda}
    \item \textbf{Manual:} \texttt{/ayuda\_medica} (Le enviará este documento en PDF).
\end{itemize}

\end{document}