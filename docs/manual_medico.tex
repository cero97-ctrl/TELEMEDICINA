\documentclass[a4paper,12pt]{article}
\usepackage[utf8]{inputenc}
\usepackage[spanish]{babel}
\usepackage{geometry}
\usepackage{hyperref}
\usepackage{xcolor}
\usepackage{titlesec}

% Configuración de márgenes
\geometry{top=3cm, bottom=3cm, left=2.5cm, right=2.5cm}

% Estilo de títulos
\titleformat{\section}{\color{blue}\normalfont\Large\bfseries}{\thesection}{1em}{}

\title{\textbf{Guía de Uso: Tu Asistente Médico de IA}}
\author{Desarrollado por: Prof. César Rodríguez}
\date{\today}

\begin{document}

\maketitle

\section*{Introducción}
Esta guía está diseñada para ayudarte a utilizar tu nuevo \textbf{Asistente de Inteligencia Artificial} a través de Telegram. Este asistente cuenta con herramientas avanzadas para investigar temas médicos, analizar documentos y responder a tus dudas mediante voz o texto, ideal para acompañarte en tu proceso de recuperación.

\section{¿Cómo empezar?}
Busca el bot en Telegram y presiona \textbf{Iniciar} o escribe \texttt{Hola}. El asistente te saludará y estará listo para ayudarte.

\section{Herramientas para tu Recuperación}

\subsection{1. Generar Reportes Médicos Detallados}
Usa el comando \texttt{/reporte} seguido del tema que te interesa. El bot buscará en fuentes confiables y redactará un informe completo con tratamientos, tiempos de recuperación y consejos.

\textbf{Ejemplo de uso:}
\begin{quote}
    \texttt{/reporte recuperación fractura de tibia y peroné}
\end{quote}
\textit{El bot te enviará un resumen al chat y guardará un documento completo para ti.}

\subsection{2. Analizar Exámenes e Informes (PDF)}
Si tienes un informe médico o resultado de laboratorio en PDF y quieres entender mejor los términos técnicos:
\begin{enumerate}
    \item Envía el archivo \textbf{PDF} al chat de Telegram.
    \item El bot lo leerá automáticamente.
    \item Te explicará en lenguaje sencillo qué dice el informe, los hallazgos principales y qué significan.
\end{enumerate}

\subsection{3. Consultas por Voz}
Si prefieres no escribir, simplemente envía una \textbf{Nota de Voz}.
\begin{itemize}
    \item Pregunta dudas como: \textit{"¿Qué ejercicios puedo hacer sentada?"} o \textit{"¿Qué alimentos tienen más calcio?"}.
    \item El bot transcribirá tu audio y te responderá por texto.
\end{itemize}

\subsection{4. Preguntas Rápidas}
Para dudas puntuales, usa el comando \texttt{/investigar}.
\begin{quote}
    \texttt{/investigar efectos secundarios del diclofenac}
\end{quote}

\section{Aviso de Seguridad}
Recuerda: \textbf{Soy una Inteligencia Artificial, no un médico.}
Toda la información que te proporciono es para fines educativos e informativos. Siempre consulta y sigue las instrucciones de tus especialistas médicos para la toma de decisiones sobre tu salud.

\vspace{2cm}
\begin{center}
    \textbf{¡Mucho ánimo en tu recuperación!}
\end{center}

\end{document}