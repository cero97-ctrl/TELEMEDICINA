\documentclass[a4paper,12pt]{article}
\usepackage[utf8]{inputenc}
\usepackage[spanish]{babel}
\usepackage{geometry}
\usepackage[hidelinks]{hyperref}
\usepackage{xcolor}
\usepackage{titlesec}
\usepackage{graphicx}

% Configuración de márgenes
\geometry{top=3cm, bottom=3cm, left=2.5cm, right=2.5cm}

% Estilo de títulos
\titleformat{\section}{\color{blue!70!black}\normalfont\Large\bfseries}{\thesection}{1em}{}
\titleformat{\subsection}{\color{blue!50!black}\normalfont\large\bfseries}{\thesubsection}{1em}{}

\title{\textbf{Manual de Usuario: Panel Médico IA}}
\author{Sistema de Telemedicina}
\date{\today}

\begin{document}

\maketitle
\tableofcontents
\newpage

\section{Introducción}
Este manual describe las funcionalidades del Asistente de Telemedicina diseñadas para el personal médico. El sistema permite el monitoreo remoto de signos vitales, gestión de pacientes y acceso a herramientas de investigación clínica.

\section{Gestión de Pacientes}

\subsection{Registro de Nuevos Pacientes}
Para ingresar un nuevo paciente al sistema de monitoreo:
\begin{itemize}
    \item \textbf{Comando:} \texttt{/nuevo\_paciente [Nombre Completo]}
    \item \textbf{Ejemplo:} \texttt{/nuevo\_paciente Juan Pérez}
    \item \textbf{Nota:} El sistema asignará automáticamente un ID único (ej. \texttt{SIM-001}). También puede asignar un ID manual: \texttt{/nuevo\_paciente SIM-005 Ana Gómez}.
\end{itemize}

\subsection{Listado de Pacientes}
Para ver todos los pacientes activos:
\begin{itemize}
    \item \textbf{Comando:} \texttt{/pacientes} o \texttt{/monitorear} (sin argumentos).
\end{itemize}

\section{Monitoreo y Telemetría}

\subsection{Ver Signos Vitales}
Para consultar el estado en tiempo real de un paciente específico:
\begin{itemize}
    \item \textbf{Comando:} \texttt{/monitorear [ID]}
    \item \textbf{Ejemplo:} \texttt{/monitorear SIM-001}
    \item \textbf{Datos mostrados:} Ritmo cardíaco, Temperatura, SpO2, Presión Arterial.
\end{itemize}

\subsection{Historial de Alertas}
Para revisar el registro de crisis pasadas:
\begin{itemize}
    \item \textbf{Comando:} \texttt{/historial\_alertas}
\end{itemize}

\section{Simulación y Control (Entrenamiento)}

\subsection{Simular Crisis}
Para fines de prueba o entrenamiento, puede inducir una alteración en los signos vitales:
\begin{itemize}
    \item \textbf{Comando:} \texttt{/simular\_crisis [ID]}
    \item \textbf{Efecto:} Eleva la frecuencia cardíaca y disminuye la saturación de oxígeno para probar el sistema de alertas.
\end{itemize}

\subsection{Estabilizar Paciente}
Para devolver los valores a la normalidad simulando un tratamiento exitoso:
\begin{itemize}
    \item \textbf{Comando:} \texttt{/estabilizar [ID]}
\end{itemize}

\subsection{Resetear Valores}
Para forzar un reinicio inmediato de los parámetros fisiológicos:
\begin{itemize}
    \item \textbf{Comando:} \texttt{/paciente\_reset [ID]}
\end{itemize}

\section{Herramientas de Investigación}

\subsection{Generar Reporte Clínico}
El agente puede investigar y redactar un informe detallado sobre una patología:
\begin{itemize}
    \item \textbf{Comando:} \texttt{/reporte [tema]}
    \item \textbf{Ejemplo:} \texttt{/reporte Diabetes Tipo 2 tratamientos modernos}
    \item \textbf{Salida:} Un archivo Markdown guardado en \texttt{docs/} y un resumen en el chat.
\end{itemize}

\subsection{Investigación Rápida}
Para consultas puntuales que requieren búsqueda en internet:
\begin{itemize}
    \item \textbf{Comando:} \texttt{/investigar [tema]}
\end{itemize}

\subsection{Análisis de Documentos}
Puede enviar un archivo PDF (historia clínica, paper) y pedir un resumen:
\begin{itemize}
    \item \textbf{Comando:} \texttt{/resumir\_archivo [nombre\_archivo]}
    \item \textbf{Nota:} El archivo debe estar en la carpeta \texttt{docs/}.
\end{itemize}

\section{Administración del Sistema}

\subsection{Estado del Servidor}
Verifique el uso de CPU, RAM y disco del servidor donde corre el agente:
\begin{itemize}
    \item \textbf{Comando:} \texttt{/status}
\end{itemize}

\subsection{Cambio de Rol}
Si necesita cambiar a la vista de paciente:
\begin{itemize}
    \item \textbf{Comando:} \texttt{/rol paciente}
\end{itemize}

\end{document}