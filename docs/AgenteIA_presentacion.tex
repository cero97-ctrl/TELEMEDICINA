\documentclass{beamer}
\usetheme{Madrid}
\usecolortheme{dolphin}
\usepackage[utf8]{inputenc}
\usepackage[spanish]{babel}
\usepackage{graphicx}
\usepackage{listings}

% Configuración de metadatos
\title{Avances del Proyecto: Agente de IA Autónomo}
\subtitle{Arquitectura, Memoria y Capacidades Multimodales}
\author{Prof. César Rodríguez \and Gemini Code Assist}
\institute{Equipo de Investigación 'Tecnología Venezolana'}
\date{\today}

\begin{document}

% Diapositiva de Título
\begin{frame}
    \titlepage
\end{frame}

% Índice
\begin{frame}{Contenido}
    \tableofcontents
\end{frame}

% Sección 1: Visión General
\section{Visión y Objetivo}
\begin{frame}{Objetivo del Proyecto}
    \begin{block}{Misión}
        Desarrollar un Agente de IA autónomo, residente en infraestructura local (Linux), capaz de apoyar en tareas de investigación académica y gestión de conocimiento para el equipo.
    \end{block}
    \vspace{0.5cm}
    \textbf{Principios de Diseño:}
    \begin{itemize}
        \item \textbf{Privacidad:} Ejecución y memoria locales.
        \item \textbf{Fiabilidad:} Separación entre razonamiento (LLM) y ejecución (Código).
        \item \textbf{Velocidad:} Uso de motores de inferencia de alta velocidad (Groq).
    \end{itemize}
\end{frame}

% Sección 2: Arquitectura
\section{Arquitectura Técnica}
\begin{frame}{Arquitectura de 3 Capas}
    Hemos implementado un diseño robusto que evita alucinaciones y errores comunes:
    \vspace{0.5cm}
    \begin{enumerate}
        \item \textbf{Capa 1: Directivas (YAML)} \\
        \textit{'El Manual de Operaciones'.} Define qué se debe hacer (SOPs).
        
        \item \textbf{Capa 2: Orquestación (LLM)} \\
        \textit{'El Cerebro'.} Toma decisiones, enruta tareas y gestiona errores. Actualmente potenciado por \textbf{Llama 3 (vía Groq)}.
        
        \item \textbf{Capa 3: Ejecución (Python)} \\
        \textit{'Las Manos'.} Scripts deterministas para tareas específicas (búsqueda web, manejo de archivos, cálculos).
    \end{enumerate}
\end{frame}

% Sección 3: Capacidades
\section{Capacidades Implementadas}
\begin{frame}{Capacidades Clave}
    \begin{itemize}
        \item \textbf{Memoria a Largo Plazo (RAG):} 
        Uso de \textit{ChromaDB} para recordar datos, preferencias y soluciones a errores entre sesiones.
        
        \item \textbf{Interfaz Telegram Completa:}
        \begin{itemize}
            \item Comandos de voz (transcripción automática).
            \item Análisis de imágenes (Visión por Computadora).
            \item Gestión remota del servidor.
        \end{itemize}
        
        \item \textbf{Velocidad Extrema:}
        Integración de \textbf{Groq} logrando respuestas en \textbf{0.6 segundos} (vs 2.6s de modelos estándar).
        
        \item \textbf{Sistema de Respaldo (Fallback):}
        Si Groq falla, el sistema cambia automáticamente a Google Gemini sin interrupción.
    \end{itemize}
\end{frame}

% Sección 4: Demo y Comandos
\section{Interacción y Demo}
\begin{frame}{Comandos Disponibles}
    El agente responde a comandos naturales y estructurados:
    \vspace{0.3cm}
    \begin{description}
        \item[/investigar] Realiza búsquedas en la web y genera resúmenes.
        \item[/recordar] Guarda información en la base de datos vectorial.
        \item[/status] Reporte de salud del servidor (CPU, RAM, Disco).
        \item[/modo] Cambio de personalidad (Serio, Sarcástico, Pirata...).
        \item[/reiniciar] Limpia el contexto y restablece la identidad.
    \end{description}
\end{frame}

% Conclusión
\section{Próximos Pasos}
\begin{frame}{Hoja de Ruta}
    \begin{itemize}
        \item Ampliación de herramientas de ejecución (Sandbox de código).
        \item Integración con calendarios y herramientas de productividad.
        \item Refinamiento del sistema de auto-curación de errores.
    \end{itemize}
    \vspace{1cm}
    \centering \Large \textbf{¡Gracias!}
\end{frame}

\end{document}